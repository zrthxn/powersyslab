\documentclass[a4paper,12pt]{article}

\usepackage[utf8]{inputenc}
\usepackage[T1]{fontenc}
\usepackage{a4}
\usepackage{lipsum}
\usepackage{graphicx}
\usepackage{float}
\usepackage{listings}
\usepackage{color}
\usepackage{hyperref}
\usepackage{cite}
\usepackage{textgreek}
\usepackage{amsfonts}
\usepackage{amsmath}

\usepackage[margin=1in]{geometry}

\definecolor{dkgreen}{rgb}{0,0.6,0}
\definecolor{gray}{rgb}{0.5,0.5,0.5}
\definecolor{mauve}{rgb}{0.58,0,0.82}

\lstset{frame=tb,
  language=matlab,
  aboveskip=5mm,
  belowskip=5mm,
  showstringspaces=false,
  columns=flexible,
  basicstyle={\small\ttfamily},
  numberstyle=\tiny\color{gray},
  keywordstyle=\color{blue},
  commentstyle=\color{dkgreen},
  stringstyle=\color{mauve},
  breaklines=true,
  breakatwhitespace=true,
  tabsize=2
}

\title{
  {\Huge \bf Power Systems Lab}\\
  \vspace{0.25in}

  {\bf Experiment 4}\\
  Laboratory Report
  \vspace{1in}
}
\author{
  \bf Syed Alisamar Husain, 17BEE012\\
  B.Tech Electrical Engg, 8th Semester
}

\begin{document}
  \begin{titlepage}
    \maketitle
    \vspace*{\fill}
    \begin{center}
      {\bfseries Department of Electrical Engineering} \\
      Jamia Millia Islamia, New Delhi
    \end{center}
    \thispagestyle{empty}
  \end{titlepage}
  
  \newpage
  \begin{center}
    \huge Experiment 4
    \vspace{0.5in}
  \end{center}

  \section{Objective}
  To write a MATLAB program to determine line currents to a Y-connected load
  by mesh analysis and by using symmetrical components.

  {\bf Let the given problem be as follows:}
  \begin{center}
    \itshape
    A balanced three phase voltage of 120 V line to neutral is applied to a 
    Y-connected load with ungrounded neutral. The three phase load consists 
    of three mutually-coupled reactances. Each phase has a series reactance of
    $Z_s = j12 \Omega$, and the mutual coupling between phases is $Z_m = j4 \Omega$.
  \end{center}
  \begin{enumerate}
    \itshape
    \item Determine the line currents by mesh analysis.    
    \item Determine the line currents by using symmetrical components.
  \end{enumerate}

  \section{Theoretical Background}
  Shown below is a 3-phase supply connected to a Y-connected load.
  Each of the voltages $V_1$, $V_2$ and $V_3$ represents the phase voltage
  which are 120 degrees out of phase with each other.
  The impedances are represented by $Z_1$, $Z_2$ and $Z_3$.

  \begin{figure}[H]
    \centering
    \includegraphics[width=5in]{img/y-load.png}
    \caption{Y-connected load with ungrounded neutral.}
    \label{yload}
  \end{figure}

    \subsection{Mesh Analysis}
    The Mesh-Current Method, also known as the Loop Current Method, 
    is quite similar to the Branch Current method in that it uses simultaneous 
    equations, Kirchhoff’s Voltage Law, and Ohm’s Law to determine unknown 
    currents in a network. It differs from the Branch Current method in that 
    it does not use Kirchhoff’s Current Law, and it is usually able to solve 
    a circuit with less unknown variables and less simultaneous equations.

    \pagebreak
    \subsection{Sequence Components}
    A set of three balanced voltages (phasors) $V_a, V_b, V_c$ is charactertzed by equal
    magnitudes and interphase differences of $120\deg$. The set is said to have a phase
    sequence $abc$ (positive sequence) if $V_b$ lags $V_a$ by $120\deg$ and $V_c$ lags 
    $V_b$ by $120\deg$.

    The three phasors can then be expressed in terms of the reference phasor $V_a$ as
    \begin{center}
      $V_a$ = $V_a$\\
      $V_b$ = $\alpha^2 V_a$\\
      $V_c$ = $\alpha V_a$\\
    \end{center}
    where the complex number operator $\alpha$ is defined as $\alpha = e^{j 120\deg}$.
    The same applies to voltages or currents.
    
    If the phase sequence is $acb$ (negative sequence), then
    \begin{center}
      $V_a$ = $V_a$\\
      $V_b$ = $\alpha V_a$\\
      $V_c$ = $\alpha^2 V_a$\\
    \end{center}

    Thus a set of balanced phasors is fully characterized by its reference phasor
    (say $V_a$) and its phase sequence (positive or negative).

    Consider now a set of three voltages (phasors) $V_a, V_b, V_c$ which in general may
    be unbalanced. According to {\bf Fortesque's theorem} the {\it three phasors can be
    described as the sum of positive, negative and zero sequence phasors}.

    \begin{center}
      $V_a = V_a^1 + V_a^2 + V_a^0$\\
      $V_b = V_b^1 + V_b^2 + V_b^0$\\
      $V_c = V_c^1 + V_c^2 + V_c^0$
    \end{center}

    The three phasor sequences (positive, negative and zero) are called the
    {\bf symmetrical components} of the original phasors. These equations 
    can be expressed in the matrix form
    
    \begin{center}
      \begin{math}
        \begin{bmatrix}
          V_a \\ V_b \\ V_c
        \end{bmatrix}
        =
        \begin{bmatrix}
          1 & 1        & 1 \\
          1 & \alpha^2 & \alpha \\
          1 & \alpha   & \alpha^2
        \end{bmatrix}
        \begin{bmatrix}
          V^0_a \\ V^1_b \\ V^2_c
        \end{bmatrix}
      \end{math}
    \end{center}

    \begin{center}
      $ \bf V_p = A V_s $\\
    \end{center}

    To find the sequence components, we can invert the equation
    \begin{center}
      $ \bf V_s = A^{-1} V_p $
    \end{center}

    \begin{center}
      where
      \begin{math}
        {\bf A^{-1}} =
        \dfrac{1}{3}
        \begin{bmatrix}
          1 & \alpha   & \alpha^2 \\
          1 & \alpha^2 & \alpha \\
          1 & 1        & 1
        \end{bmatrix}
      \end{math}
    \end{center}

  \pagebreak
  \section{Implementation}
  \begin{lstlisting}
    % To write a MATLAB program to determine line currents to a Y-connected 
    % load by mesh analysis and by using symmetrical components.

    % 17BEE012 - Alisamar Husain

    Vp = 120;       % 3-phase Supply Voltage
    
    Zs = 1j*12;     % Branch series reactance
    Zm = 1j*4;      % Branch mutual reactance

    % 1. Line currents by mesh analysis
    disp('1. Line currents by mesh analysis')
    Vl=sqrt(3)*Vp;

    Z = [ (Zs-Zm) -(Zs-Zm) 0
          0 (Zs-Zm) -(Zs-Zm)
          1 1 1];

    V = [ Vl*(cos(pi/6) + 1j*sin(pi/6))
          Vl*(cos(-pi/2) + 1j*sin(-pi/2))
          0];

    Iabc = Z \ V;
    disp(Iabc)

    % 2. Line currents by symmetric components
    disp('2. Line currents by symmetric components')
    a = cos(2*pi/3)+ 1j*sin(2*pi/3);

    A = [ 1 1 1; 
          1 a^2 a; 
          1 a a^2];
      
    Z012 = [  Zs+2*Zm 0 0
              0 Zs-Zm 0
              0 0 Zs-Zm];

    V012 = [0; Vp; 0];
    I012 = Z012 \ V012;

    Iabc = A * I012;
    disp(Iabc)
  \end{lstlisting}

  \section{Observations}
  \begin{figure}[H]
    \centering
    \includegraphics{img/run.png}
    \caption{Result in MATLAB}
    \label{result}
  \end{figure}
  The result of the above program with the given parameters 
  is shown in figure \ref{result}.

  \section{Result}
  The line currents for the given problem {\bf by mesh analysis} are found to be
  \begin{center}
    0.0000 -15.0000i A\\
    -12.9904 + 7.5000i A\\
    12.9904 + 7.5000i A
  \end{center}
  and line currents for {\bf by symmetric components} are found to be
  \begin{center}
    0.0000 -15.0000i A\\
    -12.9904 + 7.5000i A\\
    12.9904 + 7.5000i A
  \end{center}
  It is observed that identical values are obtained by both methods.

\end{document}